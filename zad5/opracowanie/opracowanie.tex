\documentclass[12pt]{article}
\usepackage[T1]{fontenc}
\usepackage[polish]{babel}
\usepackage[utf8]{inputenc}
\usepackage{graphicx}
\usepackage{amsmath}

\graphicspath{ {./images} }


\title{Zadanie numeryczne 5}
\author{Jakub Heczko}
\date{}

\begin{document}
\section{Wstęp:}
Głównym problemem zadania jest znalezienie optymalnego algorytmu do rozwiazania naszego układu rownań, który posiada lekką modyfikację w lewym dolnym rogu, a dokladnie w 1,2,3 miejscu w ostatnim wierszu, co sprawia, że w przeciwieństwie do pierwszego zadania, nie możemy od razu rozwiązać układu algorytmem thomasa.
\section{Optymalizacja:}
Aby móc optymalnie rozwiązać nasz układ należy odjąć pierwszy wiersz od ostatniego, w ten sposób wyzerujemy element w rogu, którego współrzędne to [N][N]. Należy również zrobić to co poprzednio czyli, przemnożyc całe wyrażenie $D_{2h}$ przez $h^{2}$, dostajemy wiec:
\newline
\begin{center}
    $D_{2h} = y_{n-1} - (h^{2} - 2)y_{n} + y_{n+1}$
\end{center}
Macierz będzie wyglądać w ten sposób i bedzie miała rozmiar N+1 na N+1:
\[
\begin{bmatrix}
    1 & 0 & 0 & 0 & 0 & \dots & 0\\
    1 & h^{2}-2 & 1 & 0 & 0 & \dots & 0\\ 
    0 & 1 & h^{2}-2 & 1 & 0 & \dots & 0\\
    0 & 0 & 1 & h^{2}-2 & 1 &\dots & 0\\
    \vdots & \vdots & \vdots & \ddots & \ddots & \ddots & \vdots\\
    0 & 0 & 0 & \hdots & 1 & h^{2} & 1\\
    1 & 0 & 0 & \hdots & 0 & 1 & -2
\end{bmatrix}
*
\begin{bmatrix}
    y_{0}\\
    y_{1}\\
    y_{2}\\
    y_{3}\\
    \vdots\\
    y_{n-1}\\
    y_{n}
\end{bmatrix}
=
\begin{bmatrix}
    1\\
    0\\
    0\\
    0\\
    \vdots\\
    0\\
    0
\end{bmatrix}
\]
Następnie po odjęciu wierszy dostajemy: 
\[
\begin{bmatrix}
    1 & 0 & 0 & 0 & 0 & \dots & 0\\
    1 & h^{2}-2 & 1 & 0 & 0 & \dots & 0\\ 
    0 & 1 & h^{2}-2 & 1 & 0 & \dots & 0\\
    0 & 0 & 1 & h^{2}-2 & 1 &\dots & 0\\
    \vdots & \vdots & \vdots & \ddots & \ddots & \ddots & \vdots\\
    0 & 0 & 0 & \hdots & 1 & h^{2}-2 & 1\\
    0 & 0 & 0 & \hdots & 0 & 1 & -2
\end{bmatrix}
*
\begin{bmatrix}
    y_{0}\\
    y_{1}\\
    y_{2}\\
    y_{3}\\
    \vdots\\
    y_{n-1}\\
    y_{n}
\end{bmatrix}
=
\begin{bmatrix}
    1\\
    0\\
    0\\
    0\\
    \vdots\\
    0\\
    -1
\end{bmatrix}
\]
Tak zooptymalizowany układ można rozwiązać algorytmem thomasa.

\section{Uzasadnienie wyboru algorytmu:}

Jak mozna poprostu zauwazyc dostajemy macierz trojdiagonalna po optymalizacji, ktora na potrzeby zadania zapisze w 3 osobnych wektorach, jeden wektor bedzie wielkosci N i bedzie reprezentowal wartości pod diagonalą, drugi wielkosci N+1 i bedzie reprezentowal wartosci diagonalne, a trzeci podobnie jak pierwszy wielkosci N-1 i bedzie reprezentowal wartodic nad doagonala. Teraz aby optymalnie obloczyc rozwoazanoa macierzy trojdiagonalnej nalezy zastosowac algorytm thomasa, ktory daje rozwiazanie w czasie O(n).
\begin{center}
W ogólności zapiszmy nad układ jako:
$a_{i}x{i-1} + b_{i}x{i} + c_{i}x{i+1} = d_{i}$ dla $i = 1,2,...,n$
\end{center}
Przy czym $a_{0} = 0$ oraz $c_{n} = 0$ czyli tak naprawde nasze skrocone wektory
Teraz załóżmy, że nasze rozwiązania można przedstawić w następujący sposób:
\begin{center}
$x_{i} = -w_{i}x_{i+1} + g_{i}$
\end{center}
A to mozna przeindeksować:
\begin{center}
$x_{i-1} = -w_{i-1}x_{i} + g_{i-1}$
\end{center}
Po odpowiednich podstawieniach pod nasze ogólne rozwiazanie $x_{i}$, dostaniemy:
\begin{center}
$w_{i} = \frac{c_{i}}{a_{i}w_{i-1}+b_{i}}$\newline\newline
$g_{i} = \frac{d_{i}-a_{i}w_{i-1}}{a_{i}w{i-1}+b_{i}}$\newline\newline
To będzie zachodzić dla $i=2,3,...,n$.
Wiemy również, że:\newline\newline
$w_{1} = \frac{c{1}}{b_{1}}$ ,
$g_{1} = \frac{d{1}}{b_{1}}$
\end{center}
Na samym końcu naszego algorytmu trzeba jeszcze uwzlędnić krok dla n, czyli:
\begin{center}
$x_{n} = g_{n}$
\end{center}
Teraz rozwiązaliśmy nasz układ w przód, musimy się jeszcze cofnąć i go rozwiązać z naszej pierwszej własności 
\begin{center}
$x_{i} = w_{i}x_{i+1}+w_{i}$ dla $i = n-1, n-2,...,1$
\end{center}


Dokladna pesymistyczna zlozonosc to O(3N)
\section{Omówienie wyników:}
Jak dalej mozna zauwazyć zmierzylem czas i względne porownanałem dla bardzo duzych danych te dwie wartości, przykładowe wyniki przedstawiam na dole:
\begin{center}
    Solved matrixA with numpy no efficent with time: 0.024635791778564453:
    \newline\newline
    Solved matrixA with thomas algorithm in time: 0.0011420249938964844:
    \newline\newline
    Porownanie wzgledne czasu LU do czasu thomasa: 21.572025052192068
\end{center}
\end{document}